\documentclass{classrep}
\usepackage[utf8]{inputenc}

\usepackage[pdftex]{color,graphicx}
\DeclareGraphicsExtensions{.pdf,.png,.jpg,.bmp,.gif}

\usepackage{mathtools}
\usepackage{amsthm}
\usepackage{amsfonts}
\usepackage{float}
\usepackage{subfig}
\usepackage{color}
\usepackage{tabularx}
\usepackage{listings}
\usepackage{indentfirst}
\usepackage{color}
\usepackage{url}
\usepackage{hyperref}
\usepackage[polish]{babel}
\usepackage{paralist}
\usepackage{indentfirst}

\hypersetup{colorlinks=false,pdfborder={0 0 0}}

\definecolor{javared}{rgb}{0.6,0,0} % for strings
\definecolor{javagreen}{rgb}{0.25,0.5,0.35} % comments
\definecolor{javapurple}{rgb}{0.5,0,0.35} % keywords
\definecolor{javadocblue}{rgb}{0.25,0.35,0.75} % javadoc

\lstset{language=Java,
basicstyle=\ttfamily,
keywordstyle=\color{javapurple}\bfseries,
stringstyle=\color{javared},
commentstyle=\color{javagreen},
morecomment=[s][\color{javadocblue}]{/**}{*/},
numbers=left,
numberstyle=\tiny\color{black},
stepnumber=2,
numbersep=10pt,
tabsize=4,
showspaces=false,
showstringspaces=false}

\studycycle{Informatyka, studia dzienne, mgr II st.}
\coursesemester{I}

\coursename{Przetwarzanie obrazu i dźwięku}
\courseyear{2011/2012}

\courseteacher{dr inż. Arkadiusz Tomczyk}
\coursegroup{środa, 8:30}

\author{
  \studentinfo{Paweł Musiał}{178726} \and
  \studentinfo{Łukasz Michalski}{178724}
}
\title{Zadanie 4:\\  \textbf {Rozpoznawanie izolowanych słów w sygnale mowy.}}
\svnurl{https://serce.ics.p.lodz.pl/svn/labs/poid/at_sr0830/lmpm@}

\begin{document}
\maketitle

\addtocounter{footnote}{1}

\tableofcontents
\pagebreak
\section{Cel}
Realizacja zadania polega na stworzeniu aplikacji umożliwiającej obliczanie reprezentacji sygnału audio w~postaci ciągów wektorów współczynników \textbf{MFCC} i~porównywanie ich za pomocą algorytmu \textbf{DTW}. Należy stworzyć bazę zawierającą przynajmniej 10 różnych słów (przykładowo: ,,zero'', ,,jeden'', …, ,,dziewięć'') i~wykorzystać ją do rozpoznawania słowa wypowiedzianego przez użytkownika. W~celu poprawy wyników każde słowo zawarte w bazie może być reprezentowane przez kilka wzorców (np. nagranych przez różne osoby, albo w~różnych warunkach akustycznych).

Oprócz ostatecznego wyniku rozpoznania należy zaprezentować wyniki porównań dla wszystkich słów z bazy oraz tablice $g$ (preferowana metoda – w~postaci obrazu, reprezentującego wartości $g[i, j]$ za pomocą np. odcieni szarości). Należy rozważyć metodę modyfikacji algorytmu DTW pozwalającą na dopasowanie fragmentu słowa zamiast całości. Należy zaimplementować ograniczenia globalne zgodnie z~przydzielonym wariantem, przy czym powinna też istnieć możliwość wyłączenia tych ograniczeń, tak aby ścieżka mogła mieć dowolny kształt.
\begin{itemize}
\item Ograniczenie globalne typu Sakoe and Chiba band
\item \textbf{Ograniczenie globalne typu Itakura parallelogram}
\end{itemize}
\section{Wprowadzenie}
Rozpoznawanie mowy jest klasycznym problemem przetwarzania dźwięku, dla którego w ciągu minionych dziesięcioleci zaproponowano wiele rozwiązań. Wyróżniamy tu zasadniczo problem prostszy, polegający na rozpoznawaniu izolowanych słów oraz zadanie rozpoznawania mowy ciągłej. W obu przypadkach należy przyjąć założenia odnośnie sposobu reprezentacji i parametryzacji sygnału mowy oraz odnośnie metod dopasowania danych do wzorca w sposób niezależny od czasu trwania i zmian szybkości analizowanej wypowiedzi.
\subsection{MCFF (\textit{Mel-frequency cepstral coefficients})}

\subsection{Algorytm DTW (\textit{Dynamic Time Warping})}

\section{Opis implementacji}
\subsection{MCFF (\textit{Mel-frequency cepstral coefficients})}

\subsection{Algorytm DTW (\textit{Dynamic Time Warping})}

\section{Wyniki}

\section{Dyskusja}

\section{Wnioski}

\begin{thebibliography}{1}
\bibitem{1} Fast Fourier Transform (FFT) \url{http://www.cmlab.csie.ntu.edu.tw/cml/dsp/training/coding/transform/fft.html}
\end{thebibliography}
\end{document}